\documentclass{beamer}

\mode<presentation> {
\usetheme{Madrid}
}

\beamertemplatenavigationsymbolsempty

\usepackage{pacman} % Allows you to be the best presentator ever :D

\usepackage{graphicx} % Allows including images
\usepackage{booktabs} % Allows the use of \toprule, \midrule and \bottomrule in tables
\usepackage[utf8]{inputenc}
\usepackage{float}
\usepackage{subcaption}
\usepackage{mathtools}
\usepackage{xcolor}
\usepackage{bm}

%-----------------------------------------------------------------------------
%	TITLE PAGE
%-----------------------------------------------------------------------------

\title[ML - 2019/20 - Lorenzo Palloni]{Adversarially Constrained Autoencoder Interpolations using Wasserstein Autoencoder}
\subtitle{Machine Learning}
\author{Lorenzo Palloni}
\institute[]{
    University of Florence\\
    \medskip
    \textit{lorenzo.palloni@stud.unifi.it }
}
\date{\today}

\begin{document}

\begin{frame}
\titlepage % Print the title page as the first slide
\end{frame}

%-----------------------------------------------------------------------------
%	PRESENTATION SLIDES
%-----------------------------------------------------------------------------
%   TABLE OF CONTENTS
%-----------------------------------------------------------------------------
%\begin{frame}
%\tableofcontents
%\end{frame}
%-----------------------------------------------------------------------------
%-----------------------------------------------------------------------------
%-----------------------------------------------------------------------------
\begin{frame}
\frametitle{Introduction}
  \begin{itemize}
    \item \textbf{Unsupervised Learning} context
    \item we aim to obtain "high-quality" \textbf{interpolations}
    \item interpolations example:
\end{itemize}
    ooo INSERT AN IMAGE HERE ooo
\end{frame}
%-----------------------------------------------------------------------------
\begin{frame}
\frametitle{Motivation}
  \begin{itemize}
    \item uncover underlying structure of dataset
    \item better representations $\rightarrow$ better results in other tasks
  \end{itemize}
\end{frame}
%-----------------------------------------------------------------------------
\begin{frame}
  \begin{itemize}
    \item An "high-quality" interpolation point have two characteristics:
    \medskip
      \begin{itemize}
        \item is indistinguishable from real data
        \medskip
        \item represent a semantically smooth morphing between the endpoints
      \end{itemize}
  \end{itemize}
\end{frame}
%-----------------------------------------------------------------------------
\begin{frame}
\frametitle{Techniques implemented (using pytorch)}
  \begin{itemize}
    \item ACAI (Adversarially Constrained Autoencoder Interpolations)
    \medskip
    \item WAE (Wasserstein Autoencoder)
    \medskip
    \item WWAE (Wasserstein-Wassertein Autoencoder)
  \end{itemize}
\end{frame}
%-----------------------------------------------------------------------------
\begin{frame}
\frametitle{ACAI}

\end{frame}
%-----------------------------------------------------------------------------
\begin{frame}
\frametitle{WAE}

\end{frame}
%-----------------------------------------------------------------------------
\begin{frame}
\frametitle{WWAE}

\end{frame}
%-----------------------------------------------------------------------------
\begin{frame}
\frametitle{Results on MNIST}

\end{frame}
%-----------------------------------------------------------------------------
\begin{frame}
\frametitle{Problems encoutered}

\end{frame}
%-----------------------------------------------------------------------------
\begin{frame}
\frametitle{Other applications}

\end{frame}
%-----------------------------------------------------------------------------
\begin{frame}
\frametitle{Conclusion}

\end{frame}
%-----------------------------------------------------------------------------
\begin{frame}
\frametitle{Appendix - Wasserstein distance}

\end{frame}
%-----------------------------------------------------------------------------
\begin{frame}
\frametitle{Appendix - Maximum Mean Discrepancy}

\end{frame}
%-----------------------------------------------------------------------------
\begin{frame}
\frametitle{Appendix - Inverse Multiquadratic kernel}

\end{frame}
%-----------------------------------------------------------------------------
\begin{frame}
\frametitle{Appendix - Frechèt distance between two Multinormals}

\end{frame}
%-----------------------------------------------------------------------------
%
% \begin{frame}
% \frametitle{Entity Embedding}
%     \begin{itemize}
%         \item Entity Embedding
% %        \item (during a training of a neural network model)
%         \item maps each state of a categorical variable
%     \end{itemize}
%     \begin{center}
%         $x \in \Big\{\ \text{'\textcolor{red}{red}'},\ \text{'\textcolor{green}{green}'},\ \text{'\textcolor{blue}{blue}'}\ \Big\}$
%     \end{center}
%     \begin{itemize}
%         \item in a $D$-dimensional Euclidean space
%         \item where $D \in \mathbb{N}^+$ is user-defined\footnote{$D$ might be chosen in range $[1,\ K - 1]$.}
%     \end{itemize}
%     \begin{center}
%         $x \in \Big\{\ [0.5,\ -1.2],\ [1.3,\ 0.23],\ [0.4,\ 1.1]\ \Big\}$.
%     \end{center}
% \end{frame}
% %-----------------------------------------------------------------------------
% \begin{frame}
%     \frametitle{Motivation}
%     \begin{itemize}
%         \item Let $x$ be a categorical variable with
%             \begin{center}
%                 $\mathbf{11981}$ different states.
%             \end{center}
%     \end{itemize}
%     \begin{itemize}
%         \item One Hot Encoding representation of $x$ needs
%             \begin{center}
%                 $\mathbf{11981}$-dimensional vectors.
%             \end{center}
%     \end{itemize}
%     \begin{itemize}
%         \item Entity Embedding representation of $x$ might be e.g.
%             \begin{center}
%                 $\mathbf{19}$-dimensional vectors.
%             \end{center}
%     \end{itemize}
% \medskip
%     \begin{itemize}
%         \item Explosions in dimensionality like this leads to
%             \begin{enumerate}
%                 \item drop in prediction performance (overfitting);
%                 \item computational cost in space and time.
%             \end{enumerate}
%     \end{itemize}
% \end{frame}
% %-----------------------------------------------------------------------------
% \begin{frame}
%     \frametitle{Experiments - Dataset}
%     \begin{itemize}
%         \item Dataset take from a Kaggle competition called
%         \begin{itemize}
%             \item[$\rightarrow$] Categorical Feature Encoding Challenge;
%         \end{itemize}
%         \begin{itemize}
%             \item $300k$ observations;
%             \item $23$ variables (all categorical);
%             \item binary problem ($y \in \{0, 1\}$).
%         \end{itemize}
%         \medskip
%         \item Dataset divided into
%         \begin{itemize}
%             \item 80\% $\rightarrow$ train
%             \item 20\% $\rightarrow$ test
%         \end{itemize}
%     \end{itemize}
% \end{frame}
% %-----------------------------------------------------------------------------
% \begin{frame}
%     \frametitle{Experiments - Neural Network hyperparameters}
%     \begin{itemize}
%         \item To extract the Entity Embeddings we use the following architecture:
%         \begin{enumerate}
%             \item input layer: concatenation of embedded features + other variables;
%             \item first layer: 400 hidden units and ReLU activation;
%             \item second layer: 600 hidden units and ReLU activation;
%             \item output layer: logistic function.
%         \end{enumerate}
%         \medskip
%         \item Training hyperparameters:
%         \begin{itemize}
%             \item number of epochs: 2
%             \item number of observations per mini-batch: 32
%             \item optimization algorithm: Adam\cite{adam} (default values)
%         \end{itemize}
%         \medskip
%         \item Implementation in Tensorflow\cite{tf}.
%     \end{itemize}
% \end{frame}
% %-----------------------------------------------------------------------------
% \begin{frame}
%     \frametitle{Experiments - Random Forest hyperparameters}
%     \begin{itemize}
%         \item \textbf{Random search} with $4$-fold \textbf{cross-validation} on:
%         \begin{itemize}
%             \item number of decision trees:
%             \begin{itemize}
%                 \item 125
%                 \item[$\rightarrow\bullet$] 175
%             \end{itemize}
%             \item maximum number of features used by each tree in each split:
%             \begin{itemize}
%                 \item[$\rightarrow\bullet$] 'sqrt'
%                 \item 'log2'
%             \end{itemize}
%             \item max depth of each tree:
%             \begin{itemize}
%                 \item 10
%                 \item[$\rightarrow\bullet$] 20
%                 \item None
%             \end{itemize}
%             \item minimum number of samples needed to perform a split:
%             \begin{itemize}
%                 \item 2
%                 \item[$\rightarrow\bullet$] 6
%             \end{itemize}
%         \end{itemize}
%     \end{itemize}
% \end{frame}
% %-----------------------------------------------------------------------------
% %-----------------------------------------------------------------------------
% %\begin{frame}
% %\begin{tabular}{lrrrrr}
% %\toprule
% %{} &             0 &             1 &  accuracy     \\
% %\midrule
% %precision &      0.750514 &      0.623508 &  0.731533       \\
% %recall    &      0.918996 &      0.305136 &  0.731533       \\
% %f1-score  &      0.826254 &      0.409747 &  0.731533       \\
% %support   &  41677        &  18323        &  60000          \\
% %\bottomrule
% %\end{tabular}
% %\end{frame}
% %-----------------------------------------------------------------------------
% \begin{frame}
%     \frametitle{Experiments - Random Forest Results}
%     \begin{table}
%       \centering
%       \begin{tabular}{lr}
%         \toprule
%         \multicolumn{2}{r}{AUC} \\
%         \cmidrule(r){1-2}
%         Train   & $0.9879$        \\
%         Test    & $\bm{0.6121}$ \\
%         \bottomrule
%       \end{tabular}
%       \caption{Random Forest + Entity Embeddings results.}
%     \end{table}
%     \begin{table}
%       \centering
%       \begin{tabular}{lr}
%         \toprule
%         \multicolumn{2}{r}{AUC} \\
%         \cmidrule(r){1-2}
%         Train   & $0.6818$     \\
%         Test    & $\bm{0.5640}$ \\
%         \bottomrule
%       \end{tabular}
%         \caption{Random Forest + One Hot Encoding\footnote{Variables with max 50 states used.} results.}
%     \end{table}
% \end{frame}
% %\begin{tabular}{lrrr}
% %    \begin{center}
% %        \toprule
% %{} &             0 &             1 &  accuracy     \\
% %\midrule
% %precision &      0.750514 &      0.623508 &  0.731533       \\
% %recall    &      0.918996 &      0.305136 &  0.731533       \\
% %f1-score  &      0.826254 &      0.409747 &  0.731533       \\
% %support   &  41677        &  18323        &  60000          \\
% %\bottomrule
% %\end{tabular}
% %\end{frame}
%-----------------------------------------------------------------------------
\begin{frame}
\frametitle{Conclusion}
\begin{itemize}
    \item \textbf{Entity Embedding} is an useful technique to put into your \textbf{toolbox};
    \item in some situations can lead to a \textbf{crucial} saving in computational resources.
\end{itemize}
\end{frame}
%-----------------------------------------------------------------------------
%-----------------------------------------------------------------------------
%-----------------------------------------------------------------------------
%-----------------------------------------------------------------------------
\begingroup
\footnotesize
\begin{frame}
\frametitle{References}
\begin{thebibliography}{99}

\bibitem{guo}{Guo, C., \& Berkhahn, F. (2016). Entity embeddings of categorical variables. arXiv preprint arXiv:1604.06737.}
\bibitem{adam}{Kingma, D. P., \& Ba, J. (2014). Adam: A method for stochastic optimization. arXiv preprint arXiv:1412.6980.}
\bibitem{tf}{Abadi, M., Barham, P., Chen, J., Chen, Z., Davis, A., Dean, J., ... \& Kudlur, M. (2016). Tensorflow: A system for large-scale machine learning. In 12th {USENIX} Symposium on Operating Systems Design and Implementation ({OSDI} 16) (pp. 265-283).}

\end{thebibliography}

\end{frame}
\endgroup
%-----------------------------------------------------------------------------

\end{document}
